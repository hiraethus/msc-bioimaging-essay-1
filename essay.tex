\documentclass[12pt]{article}
\usepackage{hyperref}
\usepackage[backend=biber,style=authoryear,hyperref=true]{biblatex}
\addbibresource{essay.bib}

\title{Can computer vision in medical imaging replace doctors?} % Replace with more suitable title
\author{
        Mr Michael J Jones \\
                \em{MSc Bioinformatics and Computational Diagnostics}\\
       	Health Sciences Building\\
        Queen's University Belfast
}
\date{\today}


\begin{document}
\maketitle

\begin{abstract}
This is the paper's abstract \ldots
\end{abstract}

\section{Introduction}
In \citeyear{hamilton2014digital}, \citeauthor{hamilton2014digital} write in a review article about the success we 
will see in the world of pathology of digital pathology. It was predicted in the 1980s and the 1990s that the field of 
pathology would be revolutionised by computer driven Image Analysis (IA) becoming a routine intervention in diagnosis 
of disease \parencite{hamilton2014digital}.

While the community as yet have been underwhelmed by progress in this field, the progress made in virtual microsopy 
means that we can now scan large numbers of image data which can be analysed using quantitative digital imaging 
software \parencite{mccavigan2012digital}.

% Discuss why this is changing...

\paragraph{Outline}
This article attempts to give a short outline of whether it is conceivable for digital pathology, through specific 
application of computer vision with medical imaging, to make the role of the doctor increasingly redundant.

\section{Previous work}\label{previous work}

\section{Results}\label{results}

\section{Conclusions}\label{conclusions}

\printbibliography
\end{document}
This is never printed

\documentclass[12pt]{article}
\usepackage{hyperref}
\usepackage[backend=biber,style=authoryear,hyperref=true]{biblatex}
\addbibresource{essay.bib}

\title{Can computer vision in medical imaging replace pathologists?} % Replace with more suitable title
\author{
        Mr Michael J Jones \\
                \em{MSc Bioinformatics and Computational Diagnostics}\\
       	Health Sciences Building\\
        Queen's University Belfast
}
\date{\today}


\begin{document}
\maketitle

\begin{abstract}
This is the paper's abstract \ldots
\end{abstract}

\section{Introduction}
In \citeyear{hamilton2014digital}, \citeauthor{hamilton2014digital} write in a review article about the success we 
will see in the world of pathology of digital pathology. It was predicted in the 1980s and the 1990s that the field of 
pathology would be revolutionised by computer driven Image Analysis (IA) becoming a routine intervention in diagnosis 
of disease \parencite{hamilton2014digital}. However, this did not transpire. As an example, the method employed 
at analysing histological tissues in the fields of research, pharmacology and in the diagnosis of disease by clinicians is 
by visual inspection the tissue \parencite{kriete2005automated}.

While the community as yet have been underwhelmed by progress in this field, the progress made in virtual microsopy 
means that we can now scan large numbers of image data which can be analysed using quantitative digital imaging 
software \parencite{mccavigan2012digital}.

% Discuss why this is changing...

\paragraph{Outline}
This article attempts to give a short outline of whether it is conceivable for digital pathology, through specific 
application of computer vision with medical imaging, to make the role of the doctor increasingly redundant.

\section{The elements of histopathology}
In both the fields of research and clinical pathology, examining the morphology and immunohistochemistry (the detection 
of antigens in a cell) have been crucial.

\subsection{Staining}
Many novel and varied techniques have been developed in staining biological material in order to back up the 
field of histopathology. In the early 1940s, \citeauthor{coons1942demonstration} were the first to do this kind of 
thing by demonstrating a pneumococcal antigen in tissue by the use of a fluorescent antibody which can be detected in 
fluorescence microscopy \parencite{coons1942demonstration}.

In the field of brightfield microscopy, methods such as Enzyme labeling have become successfu. These are used where 
the detection of molecules within cells can only be achieved by finding an enzyme for which it is a substrate. These 
particular enzymes, when they come into contact with these substrates cause a `chromagenic' reaction. This means, a 
coloured substance which is insoluble in water is given off which can then be observed under a light microscope 
\parencite[Ch.~2]{buchwalow2010immunohistochemistry} .

Observing the presence of scores then gave rise to methods of scoring the presence or severity of a disease by visible 
inspection. Such scales such as the \emph{H-score} giving scores such as 0, 1+, 2+ and 3+ are inherently imprecise and, 
due to their qualitative nature are invariably imprecise \parencite[Ch.~23]{dabbs2013diagnostic}.

\section{Adoption of Digital Pathology}
According to \citeauthor{dabbs2013diagnostic}, Image Analysis has seen increasing adoption in medical research, 
however, limited adoption has been seen in clinical practice, not least due to the stringent guidelines screening 
process needed for these software to be adopted \parencite[Ch.~23]{dabbs2013diagnostic}.

While there has been great progress in digital microscopy with cytology and molecular screening techniques, it is 
pertinent to take note that the majority of research, pharmacology and clinical diagnosis is achievd via visual 
inspection alone \parencite[p.~24]{mccavigan2012digital}.

\section{Image Management}
% discuss what happens when all the data starts getting scanned (hamilton et al 2014)

\section{Stereology}
Image processing provides an automated process for deriving reproducible and quantifiable metrics of the 
morphology of tissue on a tissue microarray (TMA).

However, the field of stereology provides an insight into why this is not a silver bullet. While a high-resolution 
scan of a TMA can provide a cross-section of tissue morphology, it cannot describe the three-dimensional 
composition of the area that the tissue came from Stereology tries to solve this issue by taking multiple samples 
from, for instance a tumour, and infers the total number of events in that tumour by using statistical methods 
\parencite{webster2014whole}.

\section{Results}\label{results}

\section{Conclusions}\label{conclusions}

\printbibliography
\end{document}
This is never printed

\documentclass[12pt]{article}
\usepackage{hyperref}
\usepackage[backend=biber,style=authoryear,hyperref=true]{biblatex}
\addbibresource{essay.bib}

\title{Can computer vision in medical imaging replace clinical pathologists?} % Replace with more suitable title
\author{
        Mr Michael J Jones \\
                \em{MSc Bioinformatics and Computational Diagnostics}\\
       	Health Sciences Building\\
        Queen's University Belfast
}
\date{\today}


\begin{document}
\maketitle

\begin{abstract}
%TODO
This is the paper's abstract \ldots
\end{abstract}

\section{Introduction}
In \citeyear{hamilton2014digital}, \citeauthor{hamilton2014digital} write in a review article about the success we 
will see in the world of pathology of digital pathology. In the 1980s and the 1990s medical professionals and scientists that the field of 
pathology would be revolutionised by computer driven Image Analysis (IA). They expected IA to become a routine intervention in diagnosis 
of disease \parencite{hamilton2014digital}. However, this did not transpire. As an example, the method employed 
at analysing histological tissues in the fields of research, pharmacology and in the diagnosis of disease by clinicians is 
by visual inspection of tissue \parencite{kriete2005automated}.

The community have so far been underwhelmed by the field of digital pathology, however, the progress made in virtual microsopy 
means that we can now scan large numbers of image data which can be analysed using quantitative digital imaging 
software \parencite{mccavigan2012digital}.

This article attempts to give a short outline of whether it is conceivable for digital pathology, through specific 
application of computer vision with medical imaging, to replace clinical pathologists and provide a more quantitative 
evaluation in the diagnosis of disease.

\section{The elements of histopathology}
In both the fields of research and clinical pathology, examining the morphology and immunohistochemistry (the detection 
of antigens in a cell) have been crucial.

\subsection{Staining}
\label{sec:staining}
Researchers have developer many varied techniques  in staining biological material to back up the 
field of histopathology. In the early 1940s, \citeauthor{coons1942demonstration} were the first to do this kind of 
thing by demonstrating a pneumococcal antigen in tissue by the use of a fluorescent antibody which can be detected in 
fluorescence microscopy \parencite{coons1942demonstration}.

In the field of brightfield microscopy, methods such as Enzyme labeling have become successful. These are used where 
the detection of molecules within cells can only be achieved by finding an enzyme for which it is a substrate. These 
particular enzymes, when they come into contact with these substrates cause a `chromagenic' reaction. This means 
that the reaction emits a coloured substance which is insoluble in water which can then be observed under a light microscope 
\parencite[Ch.~2]{buchwalow2010immunohistochemistry} .

Observing the presence of scores then gave rise to methods of scoring the presence or severity of a disease by visible 
inspection. Such scales such as the \emph{H-score} giving scores such as 0, 1+, 2+ and 3+ are inherently imprecise and, 
due to their qualitative nature are invariably imprecise \parencite[Ch.~23]{dabbs2013diagnostic}.

\paragraph{Automation}
% TODO

Immunohistochemistry, while a vital part of diagnosis in the field of pathology, it can be an extremely laboroious 
task. While not necessarily conducted by a pathologists specifically, laboratory technicians could spend days 
staining tissue samples.

In the late eighties and early ninties, prototypes were made such as the immunostainer controlled by an IBM 
microcomputer running a BASIC interpreter which itself was based on a machine that was controlled with punch cards 
\parencite{mawhinney1990automated}.

% TODO expand on this
Largely, the automation of immunostaining from the late eighties to the present has resulted in the full automation. 
Immunostaining alone has not rendered the role of the clinical pathologist redundant.

\section{Adoption of Digital Pathology}
According to \citeauthor{dabbs2013diagnostic}, Image Analysis has seen increasing adoption in medical research, 
however, limited adoption has been seen in clinical practice, not least due to the stringent guidelines screening 
process needed for these software to be adopted \parencite[Ch.~23]{dabbs2013diagnostic}.

While there has been great progress in digital microscopy with cytology and molecular screening techniques, it is 
pertinent to take note that the majority of research, pharmacology and clinical diagnosis is achievd via visual 
inspection alone \parencite[p.~24]{mccavigan2012digital}.

\subsection{Compression of images}


\subsection{Image Management}
% discuss what happens when all the data starts getting scanned (hamilton et al 2014)
In the event of the widespread of digital pathology, it must be taken into consideration how digital media are stored. 
In both the fields of research and in clinical practice, the indiscrimnate scanning of TMA would lead to an archive 
of unidentifiable images.

Since 2007, \parencite{stathonikos2013going} in the Pathology Department at the University Medical Center of Utrecht 
in the Netherlands have been scanning all of the tissue slides that have been produced for the purposes of archiving 
and education. \parencite{hamilton2014digital} states that to maintain this level of data, there need to be content 
management systems as well as robust hardware used to safely store the images. As an example, scanning 500 slides 
a day requires 5 Terabytes (TB) of storage media over the course of a month \parencite{hamilton2014digital}.

%TODO expand on stuff about cloud
Cloud infrastucture also allows the transmission of images to remote collaborators \parencite{webster2014whole}. Even 
if full automation of the workflow of pathologists is achieved as it stands today, the qualitative aspects of tissue 
samples will be available for pathologists across the world, making more qualitative analyses possible.

\subsection{Image Analysis} \label{sec:imageanalysis}
Once images have been stored in a format that can be retrieved easily, Image Analysis (IA) can be applied to the 
images. As mentioned in Section~\ref{sec:staining}, traditional methods of grading tissue samples can are inaccurate. 
While Computer Aided Image Analysis (CAIA) might appear an obvious place to observe a case to show where digital 
pathology can replace the role of a pathologist, \citeauthor{dabbs2013diagnostic} states that early efforts were no 
better than the method of visual inspection and, in some cases would yield better results than visual inspection 
\parencite{dabbs2013diagnostic}.

Furthermore, the best algorithms used in investigative labs employ the use of machine learning which require training 
data. In these cases, the expertise of pathologists is required to provide the necessary training data for these 
algorithms to perform optimally \parencite{webster2014whole}.

\paragraph{Available packages}
To date, there are a number of software packages for image analysis that target the field of histopathology and, more 
generally, analysis of human tissue. A number of these are visible in Table~\ref{tab:software}.

\begin{table}
\label{tab:software}
\caption{A table listing Free/Open Source Software (FOSS) and Commercial software used for stuff.}
\begin{tabular}{|  l | l |}
	\hline
	Free/Open Source & (FOSS) / Commercial \\ \hline
	ImageJ & Genoptix AQUA Technology \\
	ImmunoRatio & Genie (GENetic Imagery Exploitation) \\
	ImmunoMembrane & Aperio \\
	& DEFINIENS TissueStudio \\
	& INform from Caliper/Perkin Elmer \\ \hline
\end{tabular}

\end{table}

\paragraph{Free and Open Source software}
\citeauthor{abramoff2004image} discusses the benefits of using open source software in a scientific environment, in 
particular, ImageJ. The open source community drives the development of ImageJ by both develop and use the software extensivly. The 
authors of this article present the idea that when a novel piece of functionality is to be implemented or fixed then the 
community can be asked for assistance. However, the example used by \citeauthor{abramoff2004image} shows how 
it took two days for support for images five gigabytes in size in spite of community members suggesting a fix had been 
provided within one day which was subsequently shown to be broken \parencite{abramoff2004image}.

\paragraph{Commercial software}
While the dynamic nature of this Free/Open Source software can lead to innovation that will rapidly expand on the 
functionality of these tools, an attribute which can be very appealing to the research community, it is unlikely that
this kind of software might threaten the practise of clinical pathology due the the brittle nature of the software and
indicates that clinical practise should be reapeatable and codified.

It remains for commercial entities to contribute to the field of digital pathology that could result in the automation of
clinical practice. Given the size of the US market in terms of healthcare, it is appropriate to look to the FDA (the
United States Food and Drug Administration) to see how these kinds of products would be regulated to gain some
inside into how likely they will be adopted in the near future.

Software released by these entities (as well as hardware) are scrutinised under the Medical Device Classification
Procedure by the FDA \parencite{fdadevices} to ensure there is evidence to prove that the software is safe as well as providing
proof that the software functions as specified.

\paragraph{Shortcomings of Image Analysis}
\citeauthor{taylor2006quantification} indicates that some of the methods that are used in image processing, 
while others convenient, have their limitations. For instance, immunohistochemistry can be used to apply multiple 
stain types to one piece of tissue to, for instance, measure the presence of different antibodies.

Varying levels of success can be observed in separating out these stains in scanned images for quantitative analysis. 
In terms of brightfield imagery, unmixing of diaminobenzidine (DAB), or brown staining, from haematoxylin (H), or blue 
staining, has been relatively successful. When unmixing brown, red and blue stains, 
\citeauthor{taylor2006quantification} states that RGB values normally captured in TMA image do not segregate 
successfully \parencite{taylor2006quantification}.

As mentioned previously, some studies such as that conducted by \citeauthor{gavrielides2014observer} have found 
results between digital image analysis and visual observation can be comparable \parencite{gavrielides2014observer}. 
In such cases, it would be more economical to use an automated procedure.

\subsection{Stereology}
Image processing provides an automated process for deriving reproducible and quantifiable metrics of the 
morphology of tissue on a tissue microarray (TMA).

However, the field of stereology provides an insight into why this is not a silver bullet. While a high-resolution 
scan of a TMA can provide a cross-section of tissue morphology, it cannot describe the three-dimensional 
composition of the area that the tissue came from. Stereology tries to solve this issue by taking multiple samples 
from, for instance a tumour, and infers the total number of events in that tumour by using statistical methods 
\parencite{webster2014whole}.

\citeauthor{webster2014whole} also note that at present there are not many software packages that will apply 
stereological techniques to the image analysis TMA. While this would imply that the role of the clinical pathologist 
might be saved from being eradicated by automation, \citeauthor{suvarna2013bancroft} questions the utility of 
stereology in this context as it implies that the distribution of a number of events within a tumour can be described 
by a predefined distribution \parencite[p.~540, Ch.~23.]{suvarna2013bancroft}. This probabilistic approach might 
detract from the repeatbility of results in the exercise and, as such, might not be by a clinical pathologist anyway.

\section{The dangers of Automation}
If we consider that full automation of clinical pathological practice can be automated through digital pathology in 
terms of image analysis, we must also ask the question whether this is desirable.

In the first instance, software are complex feats of engineering, bringing together systems which contain many 
paths of execution and many states for them to be in. While we can marvel at the complexity of such systems, 
as time progresses, the explosion in high-level programming languages and libraries for all kinds of domains including 
image analysis means that the bar set for entry into the practise of writing na\"{i}ve software decreases.

From this it becomes tempting for those who are outside of the field of software development to write software without 
employing sound software engineering principles such as unit, component and functional testing. An investigation of a 
medical electron accelerator called the Therac-25 shows how software that is written for medical applications without 
testing can lead to near fatal consequences. In the case of the Therac-25, the software, written in assembler language 
`evolved' over a number of years and was written by only one software developer \parencite{leveson1993investigation}.

Another issue from the full automation of diagnosis in the field of clinical pathology is where systems begin to employ 
the use of machine learning capabilities. While we discussed earlier in this document (see 
Section~\ref{sec:imageanalysis}) that machine learning algorithms can be optimised by user intervention, 
\citeauthor{matthias2004responsibility} discuss the case of systems that require no human intervention whose 
`rules by which they act are not fixed' \parencite{matthias2004responsibility}. In such cases, it becomes increasingly
difficult to determine with whom responsibility lies if anything goes wrong.

\section{Conclusions}\label{conclusions}

It appears that those in charge of formulating diagnosis will not become redundant with the advent of the digital age 
of pathology, provided that they adopt new technologies and diversify their skills in line with the field. However, those 
who are not involved in the analysis such as laboratory technicians will become in less demand if we continue to see 
automation of laborious tasts such as immunohistochemical staining of samples. This automation will inevitably lead to 
more repeatability in spite of indicates that, in some cases, automated in quantitative power is no more powerful than 
manually doing the same steps.

The field of pathology is still evolving and, as such, so is the role of the pathologist. Undoubtedly, skills required to be a pathologist 
change over time. This is not to say that the field will become fully automated but that the 
laborious and menial tasks of pathology will be taken upon by automaton, allowing pathologists more time to involve 
themselves in other aspects of their job.

Whole slide imaging that will come from Digital Pathology will inevitably 
change the tools of clinical diagnosis and prognosis which may lead to one of two outcomes. In one instance, 
laboratory technicians will be required less and less as clinical pathologists are able to manage the equipment 
used in their departments more easily. Conversely, the diagnostic and prognostic power of the automaton wil lead to 
a redundancy in the requirement of clinical pathologists, leaving a number of laboratory technicians with skills in 
maintaining the tissue microarrays, immnuohistochemical staining machines and slide scanners as well as a number 
who would maintain the IT infrastructure in the storage, retrieval and dissemination of images across the world.

While we have seen automation previously in clinical pathology such as the establishing of immunostainers has not led 
to the closing of clinical pathology departments in hospitals, it remains to be seen whether digital pathology's 
contribution will be an aid or a replacement to this clinical practice. Furthermore, as more machine learning algorithms 
are applied to digital pathology, clinical practitioners must decide whether they are appropriate and that there is 
accountability for the behaviour of software and equipment.

\printbibliography
\end{document}
This is never printed

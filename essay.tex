\documentclass[12pt]{article}
\usepackage{hyperref}
\usepackage[table]{xcolor}
\usepackage[backend=biber,style=authoryear,hyperref=true]{biblatex}
\addbibresource{essay.bib}

\title{Can computer vision in medical imaging replace pathologists?} % Replace with more suitable title
\author{
        Mr Michael J Jones \\
                \em{MSc Bioinformatics and Computational Diagnostics}\\
       	Health Sciences Building\\
        Queen's University Belfast
}
\date{\today}


\begin{document}
\maketitle

\begin{abstract}
This is the paper's abstract \ldots
\end{abstract}

\listoftables

\section{Introduction}
In \citeyear{hamilton2014digital}, \citeauthor{hamilton2014digital} write in a review article about the success we 
will see in the world of pathology of digital pathology. It was predicted in the 1980s and the 1990s that the field of 
pathology would be revolutionised by computer driven Image Analysis (IA) becoming a routine intervention in diagnosis 
of disease \parencite{hamilton2014digital}. However, this did not transpire. As an example, the method employed 
at analysing histological tissues in the fields of research, pharmacology and in the diagnosis of disease by clinicians is 
by visual inspection the tissue \parencite{kriete2005automated}.

While the community as yet have been underwhelmed by progress in this field, the progress made in virtual microsopy 
means that we can now scan large numbers of image data which can be analysed using quantitative digital imaging 
software \parencite{mccavigan2012digital}.

% Discuss why this is changing...

\paragraph{Outline}
This article attempts to give a short outline of whether it is conceivable for digital pathology, through specific 
application of computer vision with medical imaging, to make the role of the doctor increasingly redundant.

\section{The elements of histopathology}
In both the fields of research and clinical pathology, examining the morphology and immunohistochemistry (the detection 
of antigens in a cell) have been crucial.

\subsection{Staining}
\label{sec:staining}
Many novel and varied techniques have been developed in staining biological material in order to back up the 
field of histopathology. In the early 1940s, \citeauthor{coons1942demonstration} were the first to do this kind of 
thing by demonstrating a pneumococcal antigen in tissue by the use of a fluorescent antibody which can be detected in 
fluorescence microscopy \parencite{coons1942demonstration}.

In the field of brightfield microscopy, methods such as Enzyme labeling have become successful. These are used where 
the detection of molecules within cells can only be achieved by finding an enzyme for which it is a substrate. These 
particular enzymes, when they come into contact with these substrates cause a `chromagenic' reaction. This means, a 
coloured substance which is insoluble in water is given off which can then be observed under a light microscope 
\parencite[Ch.~2]{buchwalow2010immunohistochemistry} .

Observing the presence of scores then gave rise to methods of scoring the presence or severity of a disease by visible 
inspection. Such scales such as the \emph{H-score} giving scores such as 0, 1+, 2+ and 3+ are inherently imprecise and, 
due to their qualitative nature are invariably imprecise \parencite[Ch.~23]{dabbs2013diagnostic}.

\paragraph{Automation}
% TODO

Immunohistochemistry, while a vital part of diagnosis in the field of pathology, it can be an extremely laboroious 
task. While not necessarily conducted by a pathologists specifically, laboratory technicians could spend days 
staining tissue samples.

In the late eighties and early ninties, prototypes were made such as the immunostainer controlled by an IBM 
microcomputer running a BASIC interpreter which itself was based on a machine that was controlled with punch cards 
\parencite{mawhinney1990automated}.

\section{Adoption of Digital Pathology}
According to \citeauthor{dabbs2013diagnostic}, Image Analysis has seen increasing adoption in medical research, 
however, limited adoption has been seen in clinical practice, not least due to the stringent guidelines screening 
process needed for these software to be adopted \parencite[Ch.~23]{dabbs2013diagnostic}.

While there has been great progress in digital microscopy with cytology and molecular screening techniques, it is 
pertinent to take note that the majority of research, pharmacology and clinical diagnosis is achievd via visual 
inspection alone \parencite[p.~24]{mccavigan2012digital}.

\subsection{Compression of images}


\subsection{Image Management}
% discuss what happens when all the data starts getting scanned (hamilton et al 2014)
In the event of the widespread of digital pathology, it must be taken into consideration how digital media are stored. 
In both the fields of research and in clinical practice, the indiscrimnate scanning of TMA would lead to an archive 
of unidentifiable images.

Since 2007, \parencite{stathonikos2013going} in the Pathology Department at the University Medical Center of Utrecht 
in the Netherlands have been scanning all of the tissue slides that have been produced for the purposes of archiving 
and education. \parencite{hamilton2014digital} states that to maintain this level of data, there need to be content 
management systems as well as robust hardware used to safely store the images. As an example, scanning 500 slides 
a day could require 5 Terabytes (TB) of storage media over the course of a month \parencite{hamilton2014digital}.

%TODO expand on stuff about cloud
Cloud infrastucture also allows the transmission of images to remote collaborators \parencite{webster2014whole}. Even 
if full automation of the workflow of pathologists is achieved as it stands today, the qualitative aspects of tissue 
samples will be available for pathologists across the world, making more qualitative analyses possible.

\subsection{Image Analysis}
Once images have been stored in a format that can be retrieved easily, Image Analysis (IA) can be applied to the 
images. As mentioned in Section~\ref{sec:staining}, traditional methods of grading tissue samples can are inaccurate. 
While Computer Aided Image Analysis (CAIA) might appear an obvious place to observe a case to show where digital 
pathology can replace the role of a pathologist, \citeauthor{dabbs2013diagnostic} states that early efforts were no 
better than the method of visual inspection and, in some cases would yield better results than visual inspection 
\parencite{dabbs2013diagnostic}.

%TODO more
Furthermore, the best algorithms used in investigative labs employ the use of machine learning which require training 
data. In these cases, the expertise of pathologists would be required to provide the necessary training data for these 
algorithms to perform optimally \parencite{webster2014whole}.

\paragraph{Available packages}
To date, there are a number of software packages for image analysis that target the field of histopathology and, more 
generally, analysis of human tissue. A number of these are visible in Table~\ref{tab:software}.

\begin{table}
\label{tab:software}
\begin{tabular}{|  l | l |}
	\hline
	Free/Open Source & (FOSS) / Commercial \\ \hline
	ImageJ & Genoptix AQUA Technology \\
	ImmunoRatio & Genie (GENetic Imagery Exploitation) \\
	ImmunoMembrane & Aperio \\
	& DEFINIENS TissueStudio \\
	& INform from Caliper/Perkin Elmer \\ \hline
\end{tabular}
\caption{A table listing Free/Open Source Software (FOSS) and Commercial software used for stuff.}
\end{table}

\citeauthor{abramoff2004image} discusses the benefits of using open source software in a scientific environment, in 
particular, ImageJ. ImageJ's development is driven by a community who both develop and use the software extensivly. The 
authors of this article present the idea that when a novel piece of functionality is to be implemented or fixed then the 
community can be asked for assistance. However, the example used by \citeauthor{abramoff2004image} shows how 
it took two days for support for images five gigabytes in size in spite of community members suggesting a fix had been 
provided within one day which was subsequently shown to be broken \parencite{abramoff2004image}.

While the very dynamic nature of this Free/Open Source software can lead to innovation that will rapidly expand on the 
functionality of these tools, it is unlikely that this kind of software would threaten the practise of clinical 
pathology due the the brittle nature of the software and the fact that clinical practise should be reapeatable and 
codified.

\subsection{Stereology}
Image processing provides an automated process for deriving reproducible and quantifiable metrics of the 
morphology of tissue on a tissue microarray (TMA).

However, the field of stereology provides an insight into why this is not a silver bullet. While a high-resolution 
scan of a TMA can provide a cross-section of tissue morphology, it cannot describe the three-dimensional 
composition of the area that the tissue came from. Stereology tries to solve this issue by taking multiple samples 
from, for instance a tumour, and infers the total number of events in that tumour by using statistical methods 
\parencite{webster2014whole}.

\citeauthor{webster2014whole} also note that at present there are not many software packages that will apply 
stereological techniques to the image analysis TMA. While this would imply that the role of the clinical pathologist 
might be saved from being eradicated by automation, \citeauthor{suvarna2013bancroft} questions the utility of 
stereology in this context as it implies that the distribution of a number of events within a tumour can be described 
by a predefined distribution \parencite[p.~540, Ch.~23.]{suvarna2013bancroft}. This probabilistic approach might 
detract from the repeatbility of results in the exercise and, as such, might not be by a clinical pathologist anyway.

\section{Results}\label{results}

\section{Conclusions}\label{conclusions}

It appears that those in charge of formulating diagnosis will not become redundant with the advent of the digital age 
of pathology, provided that they adopt new technologies and diversify their skills in line with the field. However, those 
who are not involved in the analysis such as laboratory technicians will become in less demand if we continue to see 
automation of laborious tastks such as immunohistochemical staining of samples.

The field of pathology is still evolving and, as such, so is the role of the pathologist. Undoubtedly, the role of the 
pathologist today will be made redundant. This is not to say that the field will become fully automated but that the 
laborious and menial tasks of pathology will be taken upon by automaton, allowing pathologists more time to involve 
themselves in other aspects of their job.

\printbibliography
\end{document}
This is never printed
